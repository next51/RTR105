\documentclass[a4paper,11pt]{article}
% LV  pakas
\usepackage{polyglossia}

\pagestyle{empty}
\usepackage{graphicx}

% lapas izmeeri
\setlength{\textwidth}{172mm}
\setlength{\textheight}{245mm}
\setlength{\evensidemargin}{-5mm}
\setlength{\oddsidemargin}{-5mm}
\setlength{\topmargin}{-12mm}
\begin{document}
\begin{flushright}
Vārds
Uzvārds

Apliecības numurs

Grupa,
Kurss
\end{flushright}
\begin{center}
\begin{Large}
\textbf{1.Laboratorijas darbs  EIMDR}
\end{Large}

\begin{large}
\textbf{Darba nosaukums}
\end{large}
\end{center}
\begin{description}

\item \underline{Darba uzdevums:}
\begin{itemize}
\item apakšuzdevums 1

\item apakšuzdevums 2

\end{itemize}

\underline{Scenārijs:}





\begin{minipage}[c]{0.5\linewidth} 
\begin{Large}\verb=funkcija lisazu_figuras.m=\end{Large}

\begin{verbatim}
function lisazu_figuras2(phi)
%ziimee Lisazhu figuras
%izveidots 0.l.d.
t = 0:0.01:1;
f1 = 3; f2 = 4;
x = cos(2*pi*f1*t);
y = sin(2*pi*f2*t+phi);
plot(x,y)
axis('square')
\end{verbatim} 

\begin{Large}\verb=scenārijs animācijai=\end{Large}

\begin{verbatim}
for phic = 0:pi/100:2*pi
lisazu_figuras2(phic)
pause(0.1)
end


\end{verbatim}
\end{minipage}
\begin{minipage}[c]{0.5\linewidth}
\centering
\includegraphics[width=0.9\linewidth]{bilde}


\end{minipage}



\item \underline{Kvadrātiskas parabolas koeficienti:}
\begin{verbatim}
C = [0.09 9 8]
\end{verbatim}

\item\underline{Secinājumi:}

Secinājumi:
Secinājumi:
Secinājumi:
Secinājumi:
Secinājumi:
Secinājumi:
Secinājumi:
Secinājumi:
Secinājumi:
Secinājumi:
Secinājumi:
Secinājumi:
Secinājumi:
Secinājumi:
Secinājumi:
Secinājumi:
Secinājumi:
\end{description}
\end{document}